\chapter{Conclusion}

The objective of this thesis has been the design of a deep neural architecture for protein
contact prediction. Proposed architecture takes ideas from both RaptorX-Contact and PConsC4:
it relies on a deep residual convolutional network, while still being able to predict multiple
contact maps at different distance thresholds. Input features are based on target
sequences, sequence homology and evolutionary couplings, in a very similar way to PConsC4.
This enables the simplification of the prediction pipeline since GaussDCA (the only
ECA-related predictor used in PConsC4) provides accurate DCA features in a pleasingly
short amount of time.

Performance has been shown to be similar to PConsC3, while being still significantly far
from the state-of-the-art RaptorX-Contact. Protein contact prediction is a very competitive
area of research, and despite the fact that state of the art is not outperformed in the present
work, the latter enables experimentation for future work and can be reused building
more innovative approaches.

Immediate future work would focus on training the model on a larger
dataset with improved preprocessing. Indeed, despite the marginal
robustness of the model to the number of effective sequences in comparison
with DCA methods, there is still room for additional performance that can
be gained by computing statistics on much larger multiple sequence alignments.
It must be noted that most of the multiple sequence alignments of training proteins
come from the RaptorX-Property server, which provides a limited number of sequence
homologs per target (typically < 1000).
Also, now that a lot of new tertiary structure have been released after CASP12,
it becomes conceivable to create a complete dataset of more than 16 000 structures.

Finally, there is still room for research now that new deep learning techniques have
been developped. For example, a new unsupervised approach based on generative adversarial
networks would solve the bottleneck issues related to the quality of the evolutionary
coupling analysis. Generative adversarial networks have already been considered in
the framework of dihedral angle prediction~\cite{kim2018dihedral}
for example. Also, the design of new end-to-end models~\cite{alquraishi2019end} for direct
structure prediction would help bypassing the two-stage prediction methods that are currently in 
use on all structure prediction servers, and enable the evolution of structure prediction
methodology towards much simpler workflows.

\chapter{Conclusion}

The objective of this thesis has been the design of a deep neural architecture for protein
contact prediction. Proposed architecture takes ideas from both RaptorX-Contact and PConsC4:
it relies on a deep residual convolutional network, while still being able to predict multiple
contact maps at different distance thresholds. Input features are based on target
sequences, sequence homology and evolutionary couplings, in a very similar way to PConsC4.
This enables the simplification of the prediction pipeline since GaussDCA (the only
ECA-related predictor used in PConsC4) provides accurate DCA features in a pleasingly
short amount of time.

\todo{Speak about performance}
\todo{Compare performance with other models}
\todo{Indicate whether performance is good for all evaluation metrics}
\todo{State-of-the-art not improved (because very competitive), but enables expermimentation}
\todo{GAN approach}

Immediate future work would focus on training the model on a larger
dataset with improved preprocessing. Indeed, despite the marginal
robustness of the model to the number of effective sequences in comparison
with DCA methods, there is still room for additional performance that can
be gained by computing statistics on much larger multiple sequence alignments.
\todo{RaptorX-Property}
Also, now that a lot of new tertiary structure have been released after CASP12,
it becomes conceivable to create a complete dataset of more than 16 000 structures.

\todo{Potential improvements due to new DL techniques}

\todo{As of drawing this conclusion, ... TODO: current work}

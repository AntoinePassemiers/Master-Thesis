\section{Evaluation} \label{evaluationmetrics}

    Protein contact maps are imbalanced by nature: they contain very few residue contacts compared to their number of residue pairs.
    $L$ being the number of residues in a protein, the number of residue contacts increases linearly with $L$ while the number of residue pairs
    increases quadratically~\cite{OLMEA1997S25}. This is important because one can evaluate a model only on the $L$ (or even less than $L$) predicted
    residue contacts the model is the most confident about. Such evaluation metric is called \textit{best-L/k PPV (Positive Predictive Value)}
    and can be formulated as follows:

    \begin{equation}
      \text{Best-L/k PPV} = \frac{\sum_{\underset {i-j \geq 6}{(i, j) \in B(L/k)}} C_{i, j}}{L/k}
    \end{equation}

    where $C_{i, j} \in \{0, 1\} \ \forall i, j, i-j \ge 6$ are boolean values indicating a predicted contact.
    $B(L/k)$ is the set of $L/k$ residue pairs with most confident (highest) predicted probabilities.
    Contacts under a residue distance of 6 amino acids are not considered during evaludation phase even though they are used during
    training phase.

    Best-L/k PPV can also be split into three separated metrics: short-range, medium-range and long-range contacts.
    These three types of contacts can be defined with the following sequence separations:

      \begin{itemize}
        \item Short-range contacts: 6 - 12 residue separation
        \item Medium-range contacts: 12 - 24 residue separation
        \item Long-range contacts: 24+ residue separation
      \end{itemize}

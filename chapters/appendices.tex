\chapter{3D model assessment}

    \section{Contact-assisted 3D modelling}

    This appendix describes the algorithm used to reconstruct
    proteins in three dimensions. Like GDFuzz3D~\cite{pietal2015gdfuzz3d},
    it uses graph distances to convert predicted contact maps
    in order to approximate distance maps. However, the proposed method is template-free
    and does not make use of MODELLER~\cite{modeller} as in GDFuzz3D.

    \subsection{Graph distances}

        Let's use the graph representation of contact maps as in section \ref{pcn}
        about Protein Contact Networks.
        The graph distance between two residues is defined as the length of
        the shortest path between them.
        Predicted contact maps are converted to binary adjacency matrices
        by keeping only the $4.5\,L$ top predicted contacts.

    \subsection{Approximate Euclidean distances}

        As explained in \cite{pietal2015gdfuzz3d}, there is a linear relationship
        between graph distances and real euclidean distances.
        Let $GD_{i,j}$ be the graph distance between residues $i$ and $j$. Then
        the euclidean distance $\delta(x_i, x_j)$ between the corresponding
        points $x_i$ and $x_j$ is approximated by:

        \begin{align}
            \delta(x_i, x_j) = 5.72 \times GD_{i,j}
        \end{align}

    \subsection{Gaussian restraints}

        \begin{table}[H]
            \centering
            \begin{tabular}{|l|c|c|c|c|}
                \hline
                Restraint type & Seq. sep. & Graph Distance (GD) & Mean & Standard deviation \\
                \hline
                \hline
                Intra-alpha & 1 & 1 (contact) & 3.82 & 0.35 \\
                Intra-alpha & 2 & 1 (contact) & 5.50 & 0.52 \\
                Intra-alpha & 3 & 1 (contact) & 5.33 & 0.93 \\
                Intra-alpha & 4 & 1 (contact) & 6.42 & 1.04 \\
                Intra-beta  & 1 & 1 (contact) & 3.80 & 0.28 \\
                Intra-beta  & 2 & 1 (contact) & 6.66 & 0.30 \\
                Alpha/beta  & $\ge$ 4 & 1 (contact) & 6.05 & 0.95 \\
                Helix/coil  & $\ge$ 4 & 1 (contact) & 6.60 & 0.92 \\
                Seq. sep.   & $\ge$ 4 & 1 (contact) & 3.82 & 0.39 \\
                All & $\ge$ 4 & any & 5.72 $\times$ GD & 1.34 $\times$ GD \\
                \hline
            \end{tabular}
            \captionof{table}{Gaussian restraints present in the 3D model}
            \label{restraints}
        \end{table}

        The set of points $X$ that best satisfies Gaussian restraints is simply
        obtained by log-likelihood maximization:
        \begin{align}
            \hat{X} & = \text{argmax}_{X} \sum\limits_{i < j ,\, (\mu_{i,j}, \sigma_{i,j}) \in R}
                \Bigg(\frac{\delta(x_i, x_j) - \mu_{i,j}}{\sigma_{i,j}}\Bigg)^2
        \end{align}

    \subsection{Evolutionary algorithm}


\section{Evaluation metrics}

\begin{align}
    \text{TM-score}(X^{(target)}, X^{(aligned)}) = \text{max} \Bigg[ \frac{1}{L} \sum\limits_{i=1}^L 
        \frac{1}{1 + \Big(\frac{\delta(x_i^{(target)}, x_i^{(aligned)})}{\delta_0}\Big)^2} \Bigg]
\end{align}

where $\delta_0 = 1.24 \sqrt[3]{L - 15} - 1.8$.


\begin{align*}
    P(x) & = R^X_{\phi} R^Y_{\psi} R^Z_{\theta} x + b \\
    & =
    \begin{pmatrix}
    1 & 0 & 0 \\
    0 & \cos{\phi} & -\sin{\phi} \\
    0 & \sin{\phi} & \cos{\phi}
    \end{pmatrix}
    \begin{pmatrix}
    \cos{\psi} & 0 & \sin{\psi} \\
    0 & 1 & 0 \\
    -\sin{\psi} & 0 & \cos{\psi}
    \end{pmatrix}
    \begin{pmatrix}
    \cos{\theta} & -\sin{\theta} & 0 \\
    \sin{\theta} & \cos{\theta} & 0 \\
    0 & 0 & 1
    \end{pmatrix}
    x +
    \begin{pmatrix}
    b^X \\
    b^Y \\
    b^Z
    \end{pmatrix}
\end{align*}
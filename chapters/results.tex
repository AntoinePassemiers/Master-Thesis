\chapter{Results}

\todo{Deep architectures in PCP: \cite{di2012deep}}

\section{Model evaluation on the different benchmark sets}

    

    \begin{table}[H]
        \centering
        \resizebox{\textwidth}{!}{
        \begin{tabular}{lccccccccc}
            \hline
            & & CASP11 & & & CAMEO & & & Membrane & \\
            \hline
            Method & Short & Medium & Long & Short & Medium & Long & Short & Medium & Short \\
            \hline
            \hline
            Wynona & - & - & - & - & - & - & - & - & - \\
            PconsC3 & 0.25 & 0.29 & 0.40 & 0.21 & 0.23 & 0.27 & 0.15 & 0.9 & 0.33 \\
            RaptorX-Contact & 0.28 & 0.35 & 0.55 & 0.23 & 0.28 & 0.42 & 0.16 & 0.22 & 0.47 \\
            MetaPSICOV & 0.26 & 0.31 & 0.39 & 0.22 & 0.22 & 0.28 & 0.16 & 0.21 & 0.35 \\
            PlmDCA & 0.14 & 0.16 & 0.27 & 0.11 & 0.13 & 0.19 & 0.08 & 0.11 & 0.21 \\
            PSICOV & 0.14 & 0.15 & 0.24 & 0.13 & 0.14 & 0.18 & 0.09 & 0.11 & 0.20 \\
            mfDCA & 0.13 & 0.15 & 0.22 & 0.10 & 0.11 & 0.15 & 0.09 & 0.12 & 0.24 \\
            \hline
        \end{tabular}
        }
        \captionof{table}{Best-L PPV of different methods on short,
        medium and long-range contacts. Results are shown for the three
        different benchmark sets: CASP11 targets, CAMEO proteins, and
        the benchmark set of membrane proteins.}
        \label{benchmark}
    \end{table}

    \subsection{CASP11}

        \todo{}

    \subsection{CAMEO}

        \todo{}

    \subsection{Membrane proteins}

        \todo{}

\section{Invariance to the number of homologous sequences}

\todo{tSNE: \cite{Maaten2008VisualizingDU}}

\chapter{Structure modelling} \label{metrics3d}

  As additional objective to this thesis, I designed a minimalist contact-assisted folding method
  available at: \,
  \href{https://github.com/AntoinePassemiers/GDE-GaussFold}{https://github.com/AntoinePassemiers/GDE-GaussFold}.

  \section{Protein 3D alignment and evaluation metrics}

    \begin{align}
        \text{TM-score}(X^{(target)}, X^{(aligned)}) = \max_P \Bigg[ \frac{1}{L} \sum\limits_{i=1}^L
            \frac{1}{1 + \Big(\frac{\delta(x_i^{(target)}, P(x_i^{(aligned)}))}{\delta_0}\Big)^2} \Bigg]
    \end{align}

    where $\delta_0 = 1.24 \sqrt[3]{L - 15} - 1.8$, $\delta(x_i, y_i)$ is the euclidean distance
    between residue coordinates $x_i$ and $y_i$, and $P$ is a projection that preserves

    The best alignment in 3D is found by determining the projection of $X^{(aligned)}$ that
    either maximizes the TM-score or minimizes the RMSD.
    Such a projection has 9 parameters:
    \begin{itemize}
        \item 3 boolean parameters that indicate whether to swap coordinates along
        the X, Y and Z dimensions, respectively.
        \item 3 real-valued parameters for translating coordinates along the X, Y and Z
        dimensions, respectively.
        \item 3 angles that parametrize the rotation matrices around the X, Y and Z axes,
        respectively.
    \end{itemize}

    \begin{align*}
        P(x) & = R^X_{\phi} R^Y_{\psi} R^Z_{\theta} x + b \\
        & =
        \begin{pmatrix}
        1 & 0 & 0 \\
        0 & \cos{\phi} & -\sin{\phi} \\
        0 & \sin{\phi} & \cos{\phi}
        \end{pmatrix}
        \begin{pmatrix}
        \cos{\psi} & 0 & \sin{\psi} \\
        0 & 1 & 0 \\
        -\sin{\psi} & 0 & \cos{\psi}
        \end{pmatrix}
        \begin{pmatrix}
        \cos{\theta} & -\sin{\theta} & 0 \\
        \sin{\theta} & \cos{\theta} & 0 \\
        0 & 0 & 1
        \end{pmatrix}
        x +
        \begin{pmatrix}
        b^X \\
        b^Y \\
        b^Z
        \end{pmatrix}
    \end{align*}

\section{Contact-assisted 3D modelling}

    This appendix describes the algorithm used to reconstruct
    proteins in three dimensions. Like GDFuzz3D~\cite{pietal2015gdfuzz3d},
    it uses graph distances to convert predicted contact maps
    in order to approximate distance maps. However, the proposed method is template-free
    and does not make use of MODELLER~\cite{modeller} as in GDFuzz3D.

  \subsection{Graph distances}

    Let's use the graph representation of contact maps as in section \ref{pcn}
    about Protein Contact Networks.
    The graph distance between two residues is defined as the length of
    the shortest path between them.
    Predicted contact maps are converted to binary adjacency matrices
    by keeping only the $4.5\,L$ top predicted contacts.

  \subsection{Approximate Euclidean distances}

    As explained in \cite{pietal2015gdfuzz3d}, there is a linear relationship
    between graph distances and real euclidean distances.
    Let $GD_{i,j}$ be the graph distance between residues $i$ and $j$. Then
    the euclidean distance $\dis(\mathbf{x_i},\mathbf{x_j})$ between the corresponding
    points $x_i$ and $x_j$ is approximated by:

    \begin{align}
        \dis(\mathbf{x_i},\mathbf{x_j}) = 5.72 \times GD_{i,j}
    \end{align}

    It must be noted that the error on the euclidean distance also increases
    with graph distance. For all residue pairs for which no tractable information
    is available and no protein template is available,
    this estimated distance remains the best estimator.

  \subsection{Gaussian restraints}

    Each residue pair may be associated to at most one Gaussian restraint.
    A Gaussian restraint is defined by its mean and standard deviation,
    computed empirically over a set of distances with specific properties
    like graph distances, sequence separation or secondary structure.

    As an example, residue pairs with a graph distance of one (residues are
    in contact) and a sequence separation of one have an average distance equal
    to the $C_{\alpha}-C_{\beta}$ distance (3.82 \AA{}), and a standard deviation
    equal to 0.35 \AA{}.

    \begin{table}[H]
        \centering
        \begin{tabular}{|l|c|c|c|}
            \hline
            Restraint type & Seq. sep. & Mean & Standard deviation \\
            \hline
            \hline
            Repulsion     & $\ge$ 6 & 20.00 & 120.00 \\
            Interior      & - & 5.00 & 10.00 \\
            Adjacent      & 1 & 3.81 & 0.1  \\
            Next adjacent & 2 & 5.20 & 0.55 \\
            Next adjacent & 3 & 7.00 & 0.71 \\
            Intra-alpha   & 1 & 3.82 & 0.35 \\
            Intra-alpha   & 2 & 5.50 & 0.52 \\
            Intra-alpha   & 3 & 5.33 & 0.93 \\
            Intra-alpha   & 4 & 6.42 & 1.04 \\
            Intra-beta    & 1 & 3.80 & 0.28 \\
            Intra-beta    & 2 & 6.66 & 0.30 \\
            Alpha/beta    & $\ge$ 4 & 6.05 & 0.95 \\
            Helix/coil    & $\ge$ 4 & 6.60 & 0.92 \\
            All & $\ge$ 4 & 5.72 $\times$ GD & 1.34 $\times$ GD \\
            \hline
        \end{tabular}
        \captionof{table}{Gaussian restraints present in the 3D model}
        \label{restraints}
    \end{table}

    The set of points $X$ that best satisfies Gaussian restraints is simply
    obtained by log-likelihood maximization:
    \begin{align}
        \hat{X} & = \argmax_{X} \sum\limits_{i < j ,\, (\mu_{i,j}, \sigma_{i,j}) \in R}
            \Bigg(\frac{\delta(x_i, x_j) - \mu_{i,j}}{\sigma_{i,j}}\Bigg)^2
    \end{align}

    where $R$ is the set of parameters of the Gaussian restraints. This notation is used
    because all residue pairs may not be restrained.

  \subsection{Optimization algorithm}

    \subsubsection{L-BFGS}

        L-BFGS has been used for optimizing the objective function. However, it had been
        observed that it is prone to getting stuck in local minima.
        The study of Chelvanayagam et al.~\cite{chelvanayagam1998combinatorial} suggests
        that solving the problem in a high dimensional space solves the issue (e.g. 5 dimensions).
        However, this suggests that a suitable method for projecting the solution back to the 3D space is available.
        Instead an heuristic algorithm has been used for global convergence, coupled with L-BFGS to also
        ensure local convergence.

    \subsection{Evolutionary algorithm}

        Gaussian log-likelihood is maximized by a vanilla genetic algorithm.
        The initial population is obtained by adding random noise to the coordinates
        predicted by the multidimensional scaling algorithm.
        Parent selection is done by creating two random partitions from current population
        and keeping the individuals that maximize log-likelihood in each one.
        A new individual is then created by taking each point from either the first
        or the second parent, randomly. Mutation is simulated by adding Gaussian noise
        to each point with a 50\% chance.
        Finally, the individual with lowest log-likelihood is replaced by the
        newly created individual.

        The set of hyper-parameters is composed of:
        \begin{itemize}
            \item Population size (default value is 2000)
            \item Partition size (default value is 50)
            \item Maximum number of iterations (default value is 200000)
            \item Standard deviation of mutation noise (default value is 10)
        \end{itemize}

\section{Evaluation of GDE-GaussFold}

    Performance on the PSICOV150 dataset is shown in figure \ref{gde_results}.
    Proposed method significantly outperforms FT-COMAR, while still being 
    far from state-of-the-art methods. This is largely explained by the fact
    that the method is template-free, and that high-quality templates are
    available for many of the PSICOV150 families.

    \begin{figure}[H]
      \begin{center}
        \includegraphics[width=\textwidth, keepaspectratio]{imgs/gde.png}
         \caption{TM-scores and RMSD of different folding methods including
         GDE-GaussFold, with density functions on the sides.}
        \label{gde_results}
      \end{center}
    \end{figure}

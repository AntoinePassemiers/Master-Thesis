\chapter{Introduction}

    \setcounter{page}{1}
    \vspace*{0.5cm}

    Proteins are large macromolecules in the form of chains of building blocks called amino acid residues.
    There are 20 common amino acid types, but certain proteins may contain 2 additional amino acid types, namely pyrrolysine and selenocystein
    .
    According to Anfinsen's dogma, the three-dimensional structure of a protein is uniquely determined by its underlying amino acid sequence,
    at least when observed in protein's native environment. When environmental conditions are met, a random coil (a sequence of amino acid residues
    oriented in random directions) will evolve towards the three-dimensional structure that minimizes Gibbs free energy.
    This process is called protein folding and has, however, a few exceptions.

    Protein Contact Prediction (PCP) can help determining the three-dimensional structure of proteins by limiting the search space to certain conformations
    forced by predicted contact maps. Protein structure is organized hierarchically: primary structure, secondary structure, tertiary structure
    and quaternary structure. Primary structure refers to the chemical composition of the protein, hence the sequence of amino acids present in it.
    Secondary structure indicates the presence of structures local to the amino acids themselves: these structures can generally be $\alpha$-helices
    or $\beta$-sheets. Tertiary structure contains information about the three-dimensional structure of the protein and results from interactions
    between side chains of some pairs of amino acids, such as hydrogen bonds, ionic bonds or disulfide bridges.
    Quaternary structure is specific to proteins having multiple polypeptide chains and describes the structure due to intermolecular interactions between
    these chains. PCP helps predicting the tertiary structure since three-dimensional models can be reconstructed from protein contact maps (PCM).
    Also, PCM is a more simplistic and robust description of a protein's geometry because it is invariant to rotations and translations.
    This simplification helps making deep learning methods perform well on structure prediction.

    Most PCP methods can be roughly divided into two categories:
    the ones based on Evolutionary Coupling Analysis (ECA) and the ones that infer contacts using
    supervised machine learning. In the former case, amino acid covariances are statistically modelled and the parameters of the underlying model
    are then generally optimized through log-likelihood maximization. In the second case, deep neural architectures are used to
    refine predictions made by low-level predictors such as ECA in order to generate high-quality contact maps.
    \todo{}

    Ultimately, PCP should help making \textit{ab initio} structure prediction.
    However, most recent methods rely on a whole raft of alignment and prediction tools.
    Given a protein encoded in FASTA format, ECA is only possible using a Multiple Sequence Alignment (MSA)
    of this target protein with homologuous proteins. These homologuous proteins come from the same protein family
    as the target protein, and therefore the most suitable family must be found.
    This can be done by matching the target sequence to a Hidden Markov Model (HMM) profile representing a family
    like in Pfam database~\cite{Pfam}. Once the homologuous sequences have been retrieved, they have to be aligned to
    the target sequence using an MSA tool like HHblits or HMMER. In the next step, evolutionary couplings are extracted from
    the MSA using an ECA predictor like PSICOV~\cite{doi:10.1093/bioinformatics/btr638} or plmDCA~\cite{EKEBERG2014341}.
    Eventually, predictions are gathered and refined using a deep neural architecture, necessitating the use
    of a differential programming library. These successive layers of dependencies are not making PCP a straightforward process.
    Therefore, it seems to be a natural choice to set as an objective for this thesis the development of a predictor with
    minimal requirements and performance close to state-of-the-art techniques.

    In this thesis, common state-of-the-art ECA techniques and deep learning models for PCP are going to be described.
    ECA methods comprise Direct Coupling Analysis (DCA) and Pseudo-Inverse Covariance matrices (PSICOV), which can
    both be seen as examples of graphical models.
    \todo{}

    \todo{One contact for 12 residues: \cite{kim2014one}}

    \todo{DCA are not sufficient -> Show the improvement of DL over DCA}
    \todo{Make a comparison with DL state-of-the-art architectures}
    \todo{Show DL invariance to the number of homologous sequences}

    \section{Proteins}

    	Some proteins are actually agreggations of multiple polypeptide chains.
	In that case they are known as protein complexes. A polypeptide chain 

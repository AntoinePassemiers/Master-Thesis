\section{Contact-assisted protein folding}

Static protein structure can be recovered with a high resolution solely based on a few
true residue contacts~\cite{kim2014one}. Because predicted contact maps
are fuzzy by nature and may contain false contacts, only top $\alpha \cdot L$
contacts (residue pairs associated to highest predicted probabilities)
are usually considered.
FT-COMAR~\cite{vassura2008ft}
GDFuzz3D~\cite{pietal2015gdfuzz3d} and MODELLER~\cite{modeller}
CONFOLD~\cite{adhikari2015confold}
CONFOLD2~\cite{adhikari2018confold2}
CoinFold~\cite{wang2016coinfold}
distance distributions~\cite{reese1996distance}

In the study of Chelvanayagam et al.~\cite{chelvanayagam1998combinatorial}, protein structure is modelled
in a distance geometry setting using Gaussian restraints with empirically known mean and variance.
Let $E(x_i, x_j)$ be the Euclidean distance between residues $i$ and $j$, where residues are being represented
by the center of their respective $C_{\beta}$ (or $C_{\alpha}$) atoms.
Let $D_{i, j}$ and $V_{i, j}$ be the average Euclidean distance and variance associated with the
Gaussian constraint between residues $i$ and $j$. Under the assumption of independence between
residue pairs, maximizing the log-likelihood reduces to minimizing the following objective function:
\begin{equation}
    \sum\limits_{i=1}^{n-1} \sum\limits_{j=i+1}^n \frac{(E(x_i, x_j) - D_{i, j})^2}{V_{i, j}}
\end{equation}
where $n$ is the protein size in residues. Constraints are chosen according to predicted secondary
structure and surface accessibility. By default, all residues are assumed to be separated by an average distance
of 120 \AA{}, with a high variance (120 \AA{}$^2$) to allow flexibility.
For each residue pair, these prior parameters are replaced by a more accurate constraint when sufficient
information is available. For example, residues with low surface accessibility are assumed to be separated
by a distance of 7.5 \AA{} from the center of mass of the protein, while residues with high surface accessibility
or assigned a higher average distance of 12 \AA{}. The Euclidean coordinates of the center of mass are additional
variables to be added to the model. Residues participating in the active site are considered to be near in space.
Adjacent and almost-adjacent residues are assigned low distances with very small variance since the distance
between adjacent $C_{\alpha}$ atoms is fixed but with a small variability induced by torsion angles.
Helices and strands, when predicted secondary structure is available, are modelled as well.
All residue pairs with a sequence separation $\in [2, 5]$ in alpha helices are being constrained,
and only residue paurs with a sequence separation $\in [2, 3]$ in 3/10 helices are being constrained.
Similarly, sequence separation in strands must be in the range $[2, 4]$.
Disulfide bonds are modelled by an average distance of 5.5 \AA{} and a variance of 0.2 \AA{}$^2$ between cysteine
residues. In sheets, center residues of adjacent strands are assigned an average distance of 4.54 \AA{} and a
variance of 0.1 \AA{}$^2$.
However, sheet topology and disulfide bonds are not always available and the solution suggested in the paper
is to have recourse to sheet and disulfide combinatorics
